\section{\heiti 相关工作}
\subsection{\heiti 数据定价}
较低的出售价格能吸引更多的买家,而更高的售价可以提升单件物品的利润,该如何平衡两者以获得最高收入?收入最大化,又称利润最大化问题,是传统经济学与博弈论领域已得到充分研究的问题,其一大成果便是数据定价。瓦尔拉斯均衡解决了完全竞争市场中卖家的收入最大化问题,单边拍卖也能从购买方角度提供解决方案。然而以上工作君基于边际效益,数字产品具有边际成本近乎为零的特点,其复制成本低廉的特性使得大数据
能免费传播。原则上而言,最有效的数据交易价格就是零,这使得以上的定价方案在数据市场中均不适用[14]。

已有的数据定价方案较为零散,适用于不同的场景,包括SQL查询[12][13][14]、线性统计(linear aggregate query)模型[15][16][17][18]、机器学习模型[19][20]、机器学习分类[21][22]、物联网数据等[23]。这些工作缺少利润的定量计算与优化,或者对市场的信息透明度有着强假设,因此不适用于我们的场景。



%邓的(可认证扰动):
\subsection{\heiti 代理人模式}


使用经典的委托-代理人模式,由[1]首次提出,假设数据代理人(data agent)是可信的第三方,让出售者提交数据,代理人扰动数据以达到免套利的效果。在这个机制中,数据购买方向数据拥有者的隐私进行赔偿后便可以计算添加的扰动多寡,从而对扰动的合法性发出质疑。当数据拥有者提高扰动时,能减少自己的隐私损失,同时让数据的可用性降低,因此能相应减少数据的价格,以增加销量。这就构成了一个多变量优化问题。之后有许多工作试图解决该优化问题以最大化代理人的收益,[2]的工作大多基于公开信息博弈,假设代理人知道数据拥有者的损失函数进行优化,[3]则取消了上述限制,根据之前的交易结果推断下次交易时的策略。再者,数据之间通常并非独立的,而是有一定的相关性,相关数据的隐私分析更为困难。为了解决上述问题,[5][6]引入了差分隐私技术,计算数据拥有者的隐私损失,并在时序数据市场[7]、关联子查询数据市场[8]等领域进行考察,在最大化数据代理人t的收益的基础上保护数据拥有者的隐私。

\subsection{\heiti 数据拥有者模式}
在实际操作中,一个完全可信的第三方难以寻找且代价太高[9],因此出现了由数据拥有者自行添加扰动的出售模式。[10]假设数据使用者和数据拥有者是商业联盟,并计算合适的隐私以最大化双方收益总和。然而市场中数据拥有者和数据使用者实际上是对立关系,数据拥有者有动机添加过量扰动,从而保护自己的隐私,对数据使用者而言则减少了数据可用性[9]。为了让数据拥有者能按照协议添加合适的扰动,需要为数据拥有者引入负反馈机制,即允许数据使用者对数据进行质疑。然而在数据交易市场中,数据使用者无法获得未扰动的原始数据,使得质疑困难重重。
