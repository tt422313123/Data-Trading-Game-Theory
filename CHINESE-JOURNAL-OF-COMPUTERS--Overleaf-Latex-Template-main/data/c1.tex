\section{\heiti 引言}
数据与数据驱动的服务在各领域具有广泛的运用,尤其是数据驱动的预测服务,包括股票预测、图像识别等。其具有零知识服务特性,意即使用者可以在不具备计算资源,不掌握机器学习的知识的前提下能使用预测服务。相较于传统的机器学习应用更加经济和方便。因此使用基于机器学习的预测服务市场欣欣向荣。

为了进行预测服务,服务提供者需要从数据拥有者处购买数据,使用数据训练机器 学习模型,并将模型作为服务出售。期间服务提供者需要决定购买的数据量,测试模型的质量,为机器学习模型的服务定价,以获取最高的利润。

然而在购买数据时,服务提供者不知道所购买的数据对模型的预测结果有何影响,数据拥有着也没动机公布数据质量,市场中也存在不同类型的服务购买者。除此之外,数据拥有者提供的数据可能包含隐私,不能直接用这种隐私数据训练服务模型。

为了解决数据隐私的问题,可以使用基于查询的数据出售方案进行数据交易。基于查询的数据出售方式能支持广泛的运用途径,适合不同场景下的模型训练问题,同时支持订制数据的准确性与隐私性,能让训练师有所选择,保持平衡。该技术通过在原始数据中添加扰动,使用差分隐私等手段保护数据拥有者的隐私。在基于查询的数据出售方案中,服务提供者决定所购买数据的精密度系数,数据拥有者根据精密度系数选择所添加的扰动和收取的费用,在为原始数据添加扰动后,将扰动后数据发送给服务提供者。

添加扰动的方法有两种,一种是请可信第三方添加,另一种是让数据拥有者自行添加。可信第三方难以找到,而且费用很高。倘若让数据拥有者去添加噪声的话,拥有者有动机去添加过量噪声,保护自己隐私的同时减少了服务提供者训练的模型质量,损害了服务提供者的利益。 

因此在基于查询的数据出售方案中引入了基于同态加密的认证系统。若数据拥有者提供的数据相当可疑,服务提供者可以对该数据发起认证。一旦发起认证,借助引入诚实但好奇的第三方,服务提供者便可以得知数据拥有者是否添加了过量扰动,若添加了过量扰动,数据拥有者便需要为自己的恶意行为赔偿服务提供者。若数据拥有者没有添加过量扰动,则其数据隐私遭到泄露,服务提供者需要为认证过程赔偿数据拥有者。

因此,本文首次考察了在可认证的隐私保护的,有不同种类的服务购买方的不完全信息的数据市场,主要贡献如下所示:
1、	我们是第一个考虑服务市场模型中存在恶意扰动者。
2、	我们对一个三方市场进行建模,考察其中各个角色的需求与行动,并定量分析他们的最优策略为何。
3、	我们设计了一套行之有效的市场机制,能保证数据拥有者倾向于按照协议添加扰动。同时市场机制满足个体理性、群体理性和免套利性。

